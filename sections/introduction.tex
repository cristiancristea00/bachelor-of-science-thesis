\unnumberedchapter{Introduction}

\unnumberedsection{Motivation and Applicability}

The advent of digital imagery has fundamentally transformed the way we communicate, share and interpret visual data. Images have become an indispensable part of human interaction, facilitating various applications ranging from social media and telecommunications to complex domains such as medicine, astronomy and remote sensing. However, these images are frequently compromised by degradations such as noise, blur and missing parts, limiting their utility and the accuracy of the inferences drawn from them. This provides the impetus for research into the domain of image inpainting, which seeks to restore these degraded colour images to their original, uncorrupted state.

Image inpainting, essentially, is a technique of filling in missing or corrupted parts of an image in a visually plausible manner. It is a subject of increasing significance within the fields of computer vision and machine learning due to its intricate complexity and vast potential applications. The problem of image restoration, specifically image inpainting, is indeed a complex one. Traditionally, methods such as texture synthesis and patch-based techniques were employed. However, these methods often struggled to reconstruct the degraded sections of complex images accurately. They faced challenges in preserving the global structure and coherence of the image, leading to subpar results, especially for large corrupted regions or intricate textures. The advancements in machine learning, particularly deep learning, have ushered in a promising era for image inpainting. Deep learning techniques, owing to their capability to model high-level abstractions, have shown immense potential in understanding the underlying patterns in image data.

The motivation to engage in this study is multilayered, encompassing both the practical application of the technique and the scientific challenges it presents. Image inpainting offers vast potential for real-world application. The technique is not only employed in the restoration of degraded regions, such as motion blur and missing parts but also for the purpose of removing unwanted elements from an image. The ability to remove irrelevant components while maintaining the image's integrity offers immense value across a broad spectrum of industries and fields.

Historically, the practice of image inpainting was conducted manually, a task requiring the expertise of graphic designers using sophisticated software tools like Adobe Photoshop. These conventional methods often necessitated a considerable investment of time and resources, especially in the context of high-resolution, complex images. The advent of automatic image inpainting techniques in recent years has sparked a transformation in the field. Leveraging machine learning algorithms, these methods can infer the missing or degraded parts of an image based on the contextual information of the surrounding regions, resulting in a largely streamlined and efficient process.

However, the automatic image inpainting process is not without its challenges. Algorithms capable of reconstructing intricate image structures while preserving natural textures and colours are yet to reach their pinnacle of development. Moreover, the performance of these algorithms can be heavily contingent on the image's specific characteristics and the nature of the degradation. These considerations underscore the necessity for ongoing exploration and enhancement of image inpainting methodologies.

\unnumberedsection{Objectives}

The primary goal of this project is to delve into the realm of image inpainting, focusing on the application of deep learning techniques. By constructing a system dedicated to the restoration of degraded colour images, this study is poised to present an exploration of the current landscape of the field. The emphasis of the investigation does not reside in pushing the boundaries of the discipline or aiming for an innovative leap forward. Instead, it seeks to illuminate the intricacies of the domain, presenting an understanding of some existing methodologies and challenges associated with image inpainting. Thus, the motivation for this project is rooted in the pursuit of knowledge and the aspiration to establish a solid foundation in the realm of image processing and machine learning, rather than in the drive for advancements in the field.

The objectives of this thesis are set to explore image inpainting using deep learning techniques. This project is committed to understanding and documenting the process of creating an inpainting system. The objectives form a cohesive research journey, each adding to our overall understanding of this field:
\begin{itemize}[leftmargin=1.5em]
    \setlength\itemsep{0.2cm}

    \item \textbf{Acquisition of online datasets} -- The first objective is to identify and acquire suitable online datasets. This is of foremost importance as the quality and relevance of the dataset fundamentally determine the potential accuracy and applicability of the developed model. Careful consideration will be given to ensure the selected datasets are representative of a wide array of image conditions.

    \item \textbf{Development of pre-processing workflow and pipeline} -- Next, the focus will be on the design and implementation of a robust pre-processing workflow. The importance of this stage lies in its ability to condition the data into a format suitable for subsequent processing by the neural network. The pre-processing pipeline will involve operations such as resizing, normalization and the application of synthetic degradation to better prepare the model for diverse image conditions.

    \item \textbf{Designing the neural network architectures} -- A vital objective of this thesis is the design of suitable neural network architectures for the task at hand. This will entail a careful evaluation of various Convolutional Neural Networks architectures. The choice of architecture will be guided by a detailed review of the literature, the nature of the datasets and the specific requirements of the image inpainting task.

    \item \textbf{Establishment of training and testing workflow} -- Henceforth, a workflow will be developed for the training and testing of the designed models. This involves the creation of a systematic procedure for feeding the data into the model, adjusting model parameters and validating the model's performance.

    \item \textbf{Assessment of model performance} -- The final objective is an assessment of the models' performance and thus various evaluation metrics will be employed. The outcomes will not only indicate the effectiveness of the models in the context of the chosen datasets but also provide insight into areas of potential improvement or further exploration.
\end{itemize}
